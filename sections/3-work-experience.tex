%%%%%%%%%%%%%%%%%%%%%%%%%%%%%%%%%%%%%%%%%%%%%%%
%%%%% Work Experience                     %%%%%
%%%%%%%%%%%%%%%%%%%%%%%%%%%%%%%%%%%%%%%%%%%%%%%

\en{\section[\faBriefcase]{Work Experience}}{}
\it{\section[\faBriefcase]{Esperienza Lavorativa}}{}

%{year--year}{Job title}{Employer}{City}{}{General description no longer than 1--2 lines.

\en{
  \cventry{03/2024 - 09/2024}{Curricular Internship for Bachelor’s Thesis}{\textsl{SESAR Lab}, University of Milan}{Milan,IT}{}{
    \textit{Secure Service-oriented Architectures Research Lab (SESAR Lab)}
  \begin{itemize}
  \item Analyzed existing verification and monitoring solutions in distributed environments and contributed to the design and implementation of the Assurance Engine (a \textbf{Rust}-based assurance framework for distributed systems deployed on \textbf{Kubernetes})
  \item Extended the existing framework to enable integration with \textbf{Elasticsearch}, used as the backend for data collection and querying.
  \item Developed and tested \textbf{formal} contracts, based on metrics, for the automated verification of \textbf{non-functional properties (NFP)} — including the \textbf{CIA triad}, \textbf{reliability}, \textbf{scalability}, and \textbf{performance} in distributed and microservices architectures.
  \item Implemented \textbf{REST APIs} to expose the engine's functionalities.
  \end{itemize}}
}
\it{
  \cventry{03/2024 - 09/2024}{Tirocinio Curricolare Interno per Tesi Triennale}{\textsl{SESAR Lab}, Università degli Studi di Milano}{Milano,IT}{}{
    \textit{Secure Service-oriented Architectures Research Lab (SESAR Lab)}
  \begin{itemize}
  \item Analizzato le soluzioni esistenti di verifica e monitoraggio in ambienti distribuiti e contribuito alla progettazione e all'implementazione dell'Assurance Engine (un framework di \textit{Assurance} basato su \textbf{Rust} per sistemi distribuiti e deployati su \textbf{Kubernetes})
  \item Esteso il framework esistente per consentire l'integrazione con \textbf{Elasticsearch}, utilizzato come backend per la raccolta e la interrogazione dei dati.
  \item Sviluppato e testato \textbf{contratti formali}, basati su metriche, per la verifica automatizzata di \textbf{proprietà non funzionali (PNF)} — inclusi la \textbf{triade CIA}, l'\textbf{affidabilità} (\textbf{reliability}), la \textbf{scalabilità} (\textbf{scalability}) e le \textbf{prestazioni} (\textbf{performance}) in architetture distribuite e a microservizi.
  \item Implementato \textbf{API REST} per esporre le funzionalità dell'engine.
  \end{itemize}}
}


\en{
  \cventry{01/2020 - 02/2020}{WBL: web developer}{\textsl{DaTrik Solutions}}{Casargo,IT}{}{\textit{High school work-based-learning}
  \begin{itemize}
    \item Work Based Learning (WBL): development of a web application.
    \item Technologies used: \textbf{HTML}, \textbf{JavaScript}, \textbf{jQuery}
  \end{itemize}}
}
\it{
  \cventry{01/2020 - 02/2020}{PCTO scolastico: programmatore web}{\textsl{DaTrik Solutions}}{Casargo,IT}{}{\textit{Percorsi per le Competenze Trasversali e per l'Orientamento (PCTO) durante la scuola superiore}
  \begin{itemize}
    \item Sviluppo di un applicativo web
    \item Tecnologie usate: \textbf{HTML}, \textbf{JavaScript}, \textbf{jQuery}
  \end{itemize}}
}

