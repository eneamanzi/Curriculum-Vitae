\usepackage{luacode}
\usepackage{calligra}

\begin{luacode*}
  require("lualibs")
  
  -- 1. Caricamento JSON (Fix per lualibs su Windows: usa tolua)
  local f = io.open("data.json", "r")
  local content = "{}"
  if f then
      content = f:read("*all")
      f:close()
  else
      tex.print("\\textbf{WARNING: data.json not found!}")
  end
  
  local data = utilities.json.tolua(content) or {}

  -- 2. Helper per la lingua
  function get_text(node)
      if not node then return "" end
      if type(node) == "string" then return node end
      
      -- Controlla se la macro \makeitalian è definita
      local is_italian = token.get_macro("makeitalian") ~= nil
      
      if is_italian and node.it then
          return node.it
      else
          return node.en or ""
      end
  end

  -- Helper per stampare liste puntate (itemize)
  function print_list(list_node)
      local items = get_text(list_node)
      if type(items) == "table" then
          tex.print("\\begin{itemize}")
          for _, item in ipairs(items) do
              tex.print("\\item " .. get_text(item))
          end
          tex.print("\\end{itemize}")
      else
          tex.print(items)
      end
  end

  -- 3. DATI PERSONALI
  function load_personal_data()
      local b = data.basics
      if not b then return end
      
      tex.print("\\name{" .. b.name .. "}{" .. b.surname .. "}")
      tex.print("\\title{" .. b.label .. "}")
      tex.print("\\born{" .. b.born .. "}")
      
      if b.location then
         tex.print("\\address{" .. b.location.address .. "}{}{}")
      end
      
      tex.print("\\phone[mobile]{" .. b.phone .. "}")
      tex.print("\\email{" .. b.email .. "}")
      tex.print("\\homepage{" .. b.url .. "}")
      
      if b.profiles then
         for _, p in ipairs(b.profiles) do
             if p.network:lower() == "linkedin" then tex.print("\\social[linkedin]{" .. p.username .. "}") end
             if p.network:lower() == "github"   then tex.print("\\social[github]{" .. p.username .. "}") end
             if p.network:lower() == "whatsapp" then tex.print("\\social[whatsapp]{" .. p.username .. "}") end
         end
      end
      
      if b.image then
         tex.print("\\photo[64pt][2pt]{" .. b.image .. "}")
      end
  end

  -- 4. ESPERIENZA LAVORATIVA (Work)
  function print_work()
      if not data.work then return end
      for _, w in ipairs(data.work) do
          
          tex.print("\\cventry{" .. w.startDate .. " - " .. w.endDate .. "}")
          tex.print("{" .. get_text(w.position) .. "}")
          tex.print("{" .. get_text(w.name) .. "}")      
          
          -- MODIFICA: Aggiunte parentesi tonde ( ) intorno alla location
          tex.print("{(" .. get_text(w.location) .. ")}")  
          
          tex.print("{}") 
          
          tex.print("{") 
          local summary = get_text(w.summary)
          if summary ~= "" then
              tex.print("\\textit{" .. summary .. "}")
          end
          print_list(w.highlights)
          tex.print("}")
      end
  end

  -- 5. ISTRUZIONE (Education)
  function print_education()
      if not data.education then return end
      for _, edu in ipairs(data.education) do
          
          tex.print("\\cventry{" .. edu.startDate .. " - " .. edu.endDate .. "}")
          tex.print("{" .. get_text(edu.studyType) .. " in " .. get_text(edu.area) .. "}")
          tex.print("{" .. get_text(edu.institution) .. "}") 
          
          -- MODIFICA: Aggiunte parentesi tonde ( ) intorno alla location
          tex.print("{(" .. get_text(edu.location) .. ")}")    
          
          tex.print("{}") 
          
          tex.print("{")
          if edu.thesis then
             local t_label = (token.get_macro("makeitalian") ~= nil) and "Tesi:" or "Thesis:"
             tex.print("\\textit{\\textbf{" .. t_label .. "}} \"" .. get_text(edu.thesis.title) .. "\"")
             
             if edu.thesis.supervisors then
                 local s_label = (token.get_macro("makeitalian") ~= nil) and "Supervisori:" or "Supervisors:"
                 tex.print("\\newline \\textit{\\textbf{" .. s_label .. "}} " .. edu.thesis.supervisors)
             end
             tex.print("\\\\") -- A capo
          end
          
          local score_txt = get_text(edu.score)
          if score_txt ~= "" then
             tex.print("\\textit{\\textbf{" .. score_txt .. "}}")
          end
          tex.print("}")
      end
  end

  -- 6. SKILLS & SOFT SKILLS
  function print_skills()
      if not data.skills then return end
      
      for _, s in ipairs(data.skills) do
          local name = get_text(s.name)
          
          if name == "Soft Skills" or name == "Competenze Trasversali" then
              tex.print("\\section[\\faHandsHelping]{" .. name .. "}")
              
              local keys = s.keywords
              for _, k in ipairs(keys) do
                  local text = get_text(k)
                  local label, content = text:match("^(.-):%s*(.*)")
                  
                  if label and content then
                      label = label:gsub("\\textbf{", ""):gsub("}", "")
                      tex.print("\\cvitem{" .. label .. "}{" .. content .. "}")
                  else
                      tex.print("\\cvitem{}{" .. text .. "}")
                  end
              end
          else
              local keyword_str = ""
              local keys = s.keywords
              for i, k in ipairs(keys) do
                  keyword_str = keyword_str .. get_text(k)
                  if i < #keys then keyword_str = keyword_str .. ", " end
              end
              tex.print("\\cvitem{" .. name .. "}{" .. keyword_str .. "}")
          end
      end
  end

  -- 7. LINGUE
  function print_languages()
      if not data.languages then return end
      for _, l in ipairs(data.languages) do
          local comment = ""
          if l.details then
              comment = get_text(l.details)
          end
          tex.print("\\cvitemwithcomment{" .. get_text(l.language) .. "}{" .. get_text(l.fluency) .. "}{" .. comment .. "}")
      end
  end

  -- 8. PROGETTI
  function print_projects()
      if not data.projects then return end
      for _, p in ipairs(data.projects) do
          tex.print("\\cvitem{\\textbf{" .. get_text(p.name) .. "}}")
          tex.print("{")
          tex.print("\\textit{" .. get_text(p.course) .. "} \\newline ")
          tex.print(get_text(p.description))
          print_list(p.highlights)
          tex.print("}")
      end
  end

  -- 9. GDPR / FOOTER
  function print_gdpr()
      -- Controlla se esiste il campo meta.gdpr
      if data.meta and data.meta.gdpr then
          local text = get_text(data.meta.gdpr)
          
          -- Spinge il testo in fondo alla pagina se c'è spazio
          tex.print("\\vfill") 
          
          -- Crea un piccolo paragrafo centrato
          tex.print("\\begin{center}")
          tex.print("\\footnotesize") -- Testo piccolo
          tex.print("\\textcolor{color2}{" .. text .. "}") -- Usa il colore secondario (grigio) del tema
          tex.print("\\end{center}")
      end
  end

  -- 10. FIRMA E DATA (Generata con font CALLIGRA)
  function print_signature()
      if data.basics.name and data.basics.surname then
          tex.print("\\vspace{1.5cm}") 
          
          tex.print("\\noindent")
          tex.print("\\begin{flushright}")
          
          -- Luogo e Data
          local city = ""
          if data.basics.location and data.basics.location.city then
             city = get_text(data.basics.location.city) .. ", "
          end
          tex.print("\\textit{" .. city .. "\\today} \\\\[0.5em]") 
          
          -- FIRMA CALLIGRAFICA
          -- Usiamo \huge perché il font Calligra è molto piccolo di base
          tex.print("{\\huge\\calligra " .. data.basics.name .. " " .. data.basics.surname .. "}")
          
          tex.print("\\end{flushright}")
      end
  end


\end{luacode*}

% ============================================================
% === DEFINIZIONE COMANDI LATEX ===
% ============================================================
\newcommand{\loadPersonalData}{\directlua{load_personal_data()}}
\newcommand{\printWorkExperience}{\directlua{print_work()}}
\newcommand{\printEducation}{\directlua{print_education()}}
\newcommand{\printSkills}{\directlua{print_skills()}}
\newcommand{\printLanguages}{\directlua{print_languages()}}
\newcommand{\printProjects}{\directlua{print_projects()}}
\newcommand{\printGDPR}{\directlua{print_gdpr()}}
\newcommand{\printSignature}{\directlua{print_signature()}}